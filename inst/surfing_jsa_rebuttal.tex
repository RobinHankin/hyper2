\documentclass[12pt]{article}
\usepackage{xcolor}
\begin{document}

\section*{MS596 ``Analysis of competitive surfing tournaments with generalized Bradley-Terry likelihoods" by Driver and Hankin: rebuttal}

Below, the reviewer's comments are in black, and our replies to the
issues are in \textcolor{blue}{blue}.  We have indicated changes to
the manuscript where appropriate.

Short story: we have accommodated all the comments with rewording and
clarification.  The resulting document is, we believe, stronger and
more scholarly than before and we recommend it to you.

\section*{Detailed rebuttal: Reviewer \#1}



\subsection*{Number of competitors}

To the authors,

An interesting paper submitted, thank you. My only
real concerns are your actual comments related to the sport of
competitive surfing.

Firstly, some confusion as you initially state heats involve one or
two other competitiors. However, you begin the second paragraph in the
introduction with a statement saying 'up to four surfers', so which is
it?

\textcolor{blue}{We have removed the inconsistency (and made the text
  more precise) by rephrasing the introduction which now reads:
  ``Competitive surfing involves surfers competing against one {\em or
    more} other surfers in heats\ldots''}

\subsection*{WSL not the main governing body}

Second statement is that the World Surf League is not the main
governing body of surfing competition. What about the International
Surfing Association, who organised and ran the Olympics, also run the
World Surfing games and national level competitions. The WSL really
only run professional level competition.

\textcolor{blue}{We have clarified that the WSL is the body for professional surfers}



\subsection*{Male vs female competition issues}

34 surfers are for the mens competition.

\textcolor{blue}{We have clarified that we are studying the men's
  championship}



\subsection*{Ignoring wildcards}

Why are you ignoring the "wildcards". No real justification for
this. Especially as some of the wildcards competed in more than one of
the competitions. So some form of statement to validate why you omit
the wildcards would be good.

\textcolor{blue}{See the table of results at the end of this document.
  We see that in general, a wildcard competitor participates in very
  few venues: of the 25 wildcards, 20 appear at only a single venue,
  four at two venues, and only one (Willcox) at three venues.
  Further, the wildcards perform poorly: of the 31 ranked scores, only
  5 are better than last or second to last.  Maximum likelihood
  Bradley-Terry strengths for such competitors are either exactly zero
  or very small (here $\sim 10^{-5}$) and are neither interesting from
  a surfing lore perspective, nor susceptible to meaningful
  statistical inference.\newline Quite apart from the fact that our
  focus was on the top-performing competitors, including such
  datapoints is problematic for several reasons.  Firstly, numerical
  optimization techniques tend to struggle as one increases the number
  of players: one is necessarily working with a higher-dimensional
  problem, leading to poorly defined maxima [the Jacobian becomes very
    small]; in addition, local maxima become more numerous and
  difficult to identify.\newline We have added a brief summary of this
  to the manuscript.}

\subsection*{Direct comparison on particular waves}



One other comment is based around the statements around judging in the
second last paragraph before section 1.1. Judging is based on the
heats conditions, as surfing is performed within an ever changing
environment. Therefore, to try and compare between surfers throughout
a competition is really not applicable. As each heat, and each wave is
different from the previous. So a direct comparison throughout a
competition is somewhat unfair for the competitors. As for the claim
around abilities, is this not subjective, as each performer is judged
upon their surfing prowess on that particular wave in that particular
heat.


\textcolor{blue}{Competition is indeed performed within an ever
  changing environment.  Each wave is indeed different from the
  previous.  We believe that these facts are the reason for the
  scoring system adopted in WSL tournaments.  We point out that the
  environment during any individual heat is common to all competitors
  in the water, so comparison between the competitors is fair, in the
  sense that no competitor can justifiably claim that he was placed at
  a disadvantage by wave quality.  All the competitors in the water
  have the same opportunity to catch the same waves (subject to wave
  priority rules, but these rules are fair in the same sense).  We
  would contend that a competitor's ability to choose an opportune
  wave is part of the skill of surfing, and is certainly necessary for
  tournament success.}

\textcolor{blue}{We use the word ``subjective" to mean ``personal
  aesthetic valuation"; we wished to emphasise that pursuits such as
  surfing are scored on judgements, rather than objective physical
  measures such as timings or distances as seen in, for example, 100m
  sprinting or archery.}
  
\subsection*{Brazilian preference}


Your hypothesis on Brazilians preference for reef breaking waves is
based on what?


\textcolor{blue}{We have clarified the issue and corrected an
  unfortunate typographical error.  The new manuscript reads: ``one
  suggestion (Burgess 2020, Ho 2021) is that Brazilian surfers tend to
  have skillsets that favour point and beach break wave types, readily
  accessible to these surfers' native Brazil; see (Scarfe 2003) for a
  scientific overview of surf break characteristics, and (Butt 2004)
  for a surfer's perspective''.  In short, we suggest that Brazilians
  {\em do not} like reef breaking waves.}

\subsection*{Points system affecting behaviour}


Statement on the points system not affecting competitiors
behaviour. How could you say this, as depending on the finishing spot
at each competition, the athlete is awarded a set number of points. So
a higher placing in a competition will award you a greater number of
points. Impacting upon your placing in the rankings for the subsequent
competition. And therefore, your seeding for the next competition.


\textcolor{blue}{Discussed below under ``points system"}

\subsection*{Clarification}


Other than that, an interesting paper. Just feel you need to clarify a
few of the points above to provide readers a clearer viewpoint of the
sport and understanding to the importance of your study. As I found it
quite interesting. Just some claims and anomalies related to
competitive surfing.
  

\section*{Detailed rebuttal: Reviewer \#2}


{\bf manuscript:} The World Surf League (WSL) is the main governing
  body of surfing competitions (World Surf League 2021).

{\bf comment:} Not exactly. What about ISA (world surfing games, Olympics, national
championships, regional championships).

\textcolor{blue}{{\bf response:} We have clarified that the WSL is the body for professional surfers}


\rule{0mm}{10mm}

{\bf manuscript:} The WSL conducts a
world tour in which the best ranked 34 surfers compete;

{\bf comment:} This only constitutes for male competition

\textcolor{blue}{{\bf response:} We have clarified that we are studying the men's
  championship}

\rule{0mm}{10mm}

{\bf manuscript: } in addition, at each tour venue, two "wildcard" surfers also enter the
competition, who are ignored here

{\bf comment: } Why are they ignored? Especially as some of the wildcards competed in multiple events.

\textcolor{blue}{{\bf response: } Discussed above}


\rule{0mm}{10mm}


{\bf manuscript: } Up to four surfers--the competitors--are in the water simultaneously,
watched by up to five judges.

{\bf comment: } You stated one or two other competitors above. So which is it?

\textcolor{blue}{{\bf response: } Discussed above}

\rule{0mm}{10mm}


{\bf manuscript: } However, for a particular heat, the order statistic--that is, which
competitor scored most highly, second highest, and third--is
informative about competitors abilities: the wave environment is
common to each surfer

{\bf  comment: } But judging is based on the heats conditions, as surfing is performed
within an ever changing environment. Therefore, to try and compare
between surfers throughout a competition is really not applicable. As
each heat, and each wave is different from the previous. So a direct
comparison is somewhat unfair for the competitors.

\textcolor{blue}{{\bf response: } Discussed above}

\rule{0mm}{10mm}

{\bf manuscript: }  and here we apply it to assess the supposed Brazilian preference for
reef-breaking waves.

{\bf comment: } Based on what hypothesis?

\textcolor{blue}{{\bf response: } Discussed above}


\subsection*{Points system}

Responding to ``the points system itself is intrinsically arbitrary:
the details of the points system does not affect the competitors'
behaviour but can change the overall ranking\ldots'' the referee asks:
``How could you say this, as depending on finishing spot at each
competition, the athlete is awarded a set number of points. So a
higher placing in a competition will award you a greater number of
points. Impacting upon your placing in the rankings for the subsequent
competition. And therefore, your seeding for the next competition.''

\textcolor{blue}{We agree that the original wording was confusing.
  Our point was that competitors are motivated to win, and if the
  points system awards (strictly) more points for higher ranking, then
  the competitors' behaviour is effectively driven by the ranking and
  not the points awarded.  They will be motivated to come first,
  regardless of how many points coming first is awarded (as long as
  there are more points for coming first than second etc).  }

\textcolor{blue}{
  So, for example, consider two points systems: System A awards 10
  points for a win, 7 for coming second and 1 for third; and system B
  awards 3 for a win, 2 for second and 1 for third.  Competitors will
  be equally motivated to be first, second, and third in the two
  systems, for in each system they score more points for a higher
  ranking.  But of course, if we have repeated trials (as in WSL) the
  two systems could easily result in different points-based rankings.
  We see a similar phenomenon in Formula 1 motor racing.}
 
\textcolor{blue}{ After discussion we believe that the manuscript
  would be improved by simply omitting the text under consideration.
  It now reads: ``\ldots the points system itself is intrinsically
  arbitrary: this would suggest that more formal statistical methods
  are needed for analysis of such datasets\ldots''.  Our emphasis was
  not on the disadvantages of the points systems used by WSL; we were
  more interested in analysing the results formally.
}



{\footnotesize
\begin{verbatim}
             nation wct01 wct02 wct03 wct04 wct05 wct06 wct07 wct08 wct09 wct10 wct11 points
 Ferreira       BRA     1     5    17     5    17     2    17     9     2     1     1  59740
 Medina         BRA     5     5    17    17     5     1     2     1     9     9     2  56475
 Smith          ZAF     3     3    17     5     2     9     3     5     9     2    17  49985
 Toledo         BRA     9     2     5    17     1     3     9     2    17     5    17  49145
 Andino         USA     2    17     5     2     3     3    17    17     5     5     9  46655
 Igarashi       JPN     9     9     1     9     5     5    17     9    17     3    17  40185
 Florence       HAW     3     1    17     1     5   INJ   INJ   INJ   INJ   INJ     5  37700
 Slater         USA    33     5     3     9     9     9    17     9    17     9     3  34845
 Wright         AUS     9     9    17     9    17     5     1     3    17    17    17  34780
 Flores         FRA    17     9     2    17    33    17     5    17     1    17    33  32515
 Wilson         AUS    17    17     9     3     5    17     9     5     9    17     9  31515
 Moniz          HAW     5     9    17     5    17    17     3    17     9    33     9  29525
 Bourez         FRA     9    17     9     9     9     9     9    33     9    17     5  29315
 Callinan       AUS    17     3     9     5    17     9    17    17     5    33    17  27535
 Freestone      AUS    17    17     9    17    17    17     9    17     3     5     5  27535
 Colapinto      USA    17   INJ    17    17     9    17     9     3    17     9     3  27450
 Ibelli         BRA    33    17    17     3    17    17     5    17    17     3     9  26885
 Carmichael     AUS     5    17     5    33     9    17    17     9     9     9    17  26760
 Buchan         AUS    17    17     5    33    33     5    17     5     5    17    17  25630
 Coffin         USA     5     9     9     9    17    17    17    17    17     9    17  23345
 Crisanto       BRA    17     9    17     9    33     9    17    33    17     5     9  23345
 Dora           BRA     9    17    17    17    17    17    17     5     9    17     5  22780
 Silva          BRA    17     9    17    17     9     9     9     9    33    17    17  21920
 Cardoso        BRA     9     9    17    17    17     9    17     9    17    17    17  19930
 Mendes         BRA    17    33     9    17     9    33    17    17    17     9     9  19930
 Rodrigues      BRA    17    17     3    17     9    17    33    33    17     9    33  19640
 Zietz          HAW    17   PAR    33     9    17     5    17     9    33    17    17  18300
 Duru           FRA    17    33     9    17     9    17     9    17    17    17    17  17940
 Lau            HAW    17    17    33    17    17     9    17     9     9    33    17  17940
 Bailey         AUS    17    17    33    17    17    17    17    17    17     9     9  15950
 Fioravanti     ITA    33    17     9    17   INJ   INJ   INJ   INJ     3    17    33  14455
 Andre          BRA    33    17    17    17    17    33     5    17    33    17    17  14320
 Christie       NZL    17    17    17    17    17    17    17    17    33    33     9  13960
 Morais         PRT     -     -     -    33     3    17    33     -    17    17    33  10870
 deSouza        BRA   INJ   INJ   INJ   INJ    17    17     5   INJ   INJ   INJ   INJ   8995
 Wright_M       AUS     9    17    17   INJ   INJ   INJ   INJ   INJ   INJ   INJ   INJ   7570
 Willcox        AUS     -     5    33    33     -     -     -     -     -     -     -   5275
 Lacomare       FRA     -     -     -     -     -     -     -     -     5     -     -   4745
 Heazlewood     AUS     9    17     -     -     -     -     -     -     -     -     -   4650
 Robinson       AUS     -     -     -     9     -     -     -     -     -     -     -   3320
 Vaast          FRA     -     -     -     -     -     -     9     -     -     -     -   3320
 Colapinto_C    USA     -     -     -     -     -     -     -    17     -    17     -   2660
 Herdy          BRA    17     -     -     -   INJ     -     -    33     -     -     -   1860
 Couzinet       FRA     -     -     -     -     -    33     -     -    17     -     -   1595
 Waida          INA     -     -    17     -     -     -     -     -     -     -     -   1330
 Kymerson       BRA     -     -     -     -    17     -     -     -     -     -     -   1330
 February       ZAF     -     -     -     -     -    17     -     -     -     -     -   1330
 Mamiya         HAW     -     -     -     -     -     -     -    17     -     -     -   1330
 Matson         USA     -     -     -     -     -     -     -    17     -     -     -   1330
 Schilling      USA     -     -     -     -     -     -     -    17     -     -     -   1330
 Mignot         FRA     -     -     -     -     -     -     -     -    17     -     -   1330
 Ribeiro_V      PRT     -     -     -     -     -     -     -     -     -    17     -   1330
 Blanco         PRT     -     -     -     -     -     -     -     -     -    17     -   1330
 deVault        HAW     -     -     -     -     -     -     -     -     -     -    17   1330
 Kemper         HAW     -     -     -     -     -     -     -     -     -     -    17   1330
 Mann           AUS     -    33     -     -     -     -     -     -     -     -     -    265
 Huxtable       AUS     -    33     -     -     -     -     -     -     -     -     -    265
 Ribeiro_A      BRA     -     -     -     -    33     -     -     -     -     -     -    265
 DeVries        ZAF     -     -     -     -     -    33     -     -     -     -     -    265
 Newton         HAW     -     -     -     -     -     -    33     -     -     -     -    265
 Drollet        PYF     -     -     -     -     -     -    33     -     -     -     -    265
\end{verbatim}
}
The wildcards are from Willcox onwards.

\end{document}


