\documentclass[12pt]{article}
\usepackage{xcolor}
\begin{document}

\section*{JSS4832: Generalized Plackett-Luce Likelihoods by Robin
K. S. Hankin}


Below, the reviewer's comments are in black, and the replies to the
issues are in \textcolor{blue}{blue}.  I have indicated changes to
the manuscript where appropriate.

\textcolor{blue}{Short story: I have accommodated all the comments with
rewording and clarification.}

From the editorial team:

I had a closer look at the revision of 4832 and, as far as I'm
concerned, the paper can be accepted as it is. However, I wanted to
draw your attention to the author's rebuttal file, p.7. It seems that
he doesn't agree that this paper should be published as code snippet,
as opposed to a regular paper. I'm not sure whether this was
previously communicated to the author by the editorial team. Anyway, I
think it would be good if the editorial team could clarify this issue
directly with the author.

\textcolor{blue}{Happy to classify as a code snippet if the editors
  wish, although I stand by my previous comments.  Happy either way.}


Also, I think he misunderstood my concerns about the supplemental R
code file (see bottom of p.7 in the rebuttal file). What I'm asking
for is to make the R code look nicer with proper comments, as opposed
to just pull it from the Rnw file. In addition, the code starts with


\begin{verbatim}
source("loadlibrary.R",echo=FALSE) # to avoid stupid messages
opts_chunk$set(cache=TRUE, autodep=TRUE)
\end{verbatim}

This is not reproducible. Also, the author doesn't even load the hyper2 package.

\textcolor{blue}{Fixed, code fully reproducible}

So, while I think the paper is in good shape, the code file is not. But this can be easily be fixed.

Manuscript style comments:

\begin{itemize}
  \item Please use "R>" as the command prompt, rather than ">".
\end{itemize}
    
\textcolor{blue}{fixed}

\begin{itemize}
\item The code presented in the manuscript should not contain comments
within the verbatim code. Instead the comments should be made in the
normal LaTeX text.
\end{itemize}

\textcolor{blue}{fixed}


\begin{itemize}
  \item For the code layout in R publications, we typically distinguish
input/output using Sinput/Soutput (or equivalently
CodeInput/CodeOutput). Unless there are special reasons to format it
differently, the input should use the text width (up to 76 or 77
characters) and be indented by two spaces, e.g.,

\begin{verbatim}
\begin{Sinput}
R> example_model <- lm(response ~ variable1 + variable2 + variable3,

weights = w, data = mydata)
\end{Sinput}
\end{verbatim}
\end{itemize}

\textcolor{blue}{fixed}


 As a reminder, please make sure that: \proglang, \pkg and \code have
 been used for highlighting throughout the paper (including titles and
 references), except where explicitly escaped.

\textcolor{blue}{fixed}


References:

\begin{itemize}
      \item MATLAB (not: Matlab, matlab)
guages. Hunter (2004), for example, presents results in matlab [although he works with a
\end{itemize}
\textcolor{blue}{fixed}

\begin{itemize}
\item As a reminder,
- Please make sure that all software packages are \cite{}'d properly.

- All references should be in title style.

- See FAQ for specific reference instructions.
Code:
\end{itemize}

\begin{itemize}
\item As a reminder, please make sure that the files needed to replicate
all code/examples within the manuscript are included in a standalone
replication script.
\end{itemize}
\textcolor{blue}{fixed}

\end{document}

