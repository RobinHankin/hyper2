\documentclass[article]{ajs}
\RequirePackage{amsthm,amsmath,amsfonts,amssymb}
\RequirePackage{graphicx}%% uncomment this for including figures

%%%%%%%%%%%%%%%%%%%%%%%%%%%%%%
%% declarations for jss.cls %%%%%%%%%%%%%%%%%%%%%%%%%%%%%%%%%%%%%%%%%%
%%%%%%%%%%%%%%%%%%%%%%%%%%%%%%
 
%% almost as usual
\author{Robin K. S. Hankin\,\orcidlink{0000-0001-5982-0415}\\
  University of Stirling\\ Scotland}
\title{A Simplified Plackett-Luce Likelihood Function for Rank Orders}

%% for pretty printing and a nice hypersummary also set:
\Plainauthor{Robin K. S. Hankin} %% comma-separated
\Plaintitle{A Parametrized Plackett-Luce Likelihood Function for Rank Orders}


%% an abstract and keywords
\Abstract{

Here a new simplification of the Plackett--Luce likelihood functions
for order statistics is proposed.  We consider a field of $n$ entities
that are ordered in some way [the preferred example is competitors in
  a running race, ordered by time of crossing the finishing line].  We
assign Plackett--Luce strengths $\alpha x,\alpha x^2\ldots,\alpha x^n$
where $0\leq x\leqslant 1$ is a common ratio and $\alpha$ a
normalising constant.  One entity is of particular interest to us: the
``focal entity'', or focal competitor whose strength is $\alpha x^r$
for some unknown value of $r$ with $1\leqslant r\leqslant n$.  We
consider the following inference problem: given the order statistic
for the focal competitor [first, second, etc], and the total number of
competitors, what might we infer about $x$ and $r$?  Such problems
arise when non-focal competitors' performances are censored, unknown,
or irrelevant.  }

\Keywords{keywords, comma-separated, non-capitalized, \proglang{R}}
\Plainkeywords{keywords, comma-separated, non-capitalized, R} %% without formatting
%% at least one keyword must be supplied

%% publication information
%% NOTE: Typically, this can be left commented and will be filled out by the technical editor
%% \Volume{50}
%% \Issue{9}
%% \Month{June}
%% \Year{2012}
%% \Submitdate{2012-06-04}
%% \Acceptdate{2012-06-04}
%% \setcounter{page}{1}
\Pages{1--xx}

%% The address of (at least) one author should be given
%% in the following format:
\Address{
  Robin K. S. Hankin\\
  University of Stirling\\
  Scotland\\
  E-mail: \email{hankin.robin@gmail.com}\\
  URL: \url{https://www.stir.ac.uk/people/1966824}
}
%% It is also possible to add a telephone and fax number
%% before the e-mail in the following format:
%% Telephone: +43/512/507-7103
%% Fax: +43/512/507-2851

%% for those who use Sweave please include the following line (with % symbols):
%% need no \usepackage{Sweave.sty}

%% end of declarations %%%%%%%%%%%%%%%%%%%%%%%%%%%%%%%%%%%%%%%%%%%%%%%


\begin{document}

%% include your article here, just as usual

\section{Introduction}

Ranked data arises whenever multiple entities are sorted by some
criterion; the preferred interpretation is a race in which the
sufficient statistic is the order of competitors crossing the
finishing line.  The Plackett--Luce likelihood
function~\citep{luce1959,plackett1975} is a generalization of the
Bradley--Terry model~\citep{bradley1952} in which non-negative
strengths are assigned to each competitor; without loss of generality,
if the finishing order is $1,2,\ldots,n$ and competitor $i$ has
strength $p_i\geqslant 0$, then the Plackett--Luce likelihood function
for this observation will be proportional to

\begin{equation}\label{plackettluce}
\prod_{i=1}^n\frac{p_i}{\sum_{j=i}^np_j}
\end{equation}

\noindent where the strengths are normalised so $\sum p_i=1$.
However, estimation of the strengths is somewhat difficult if $n$ is
even moderately large: even with as few as 20 competitors following
Zipf's law~\citep{zipf1949} [a typical assumption for Bradley--Terry
  strengths] and say 20 independent complete rank observations,
\cite{hankin2017_rmd} shows that maximum likelihood strengths
identify the correct ranking with a probability of essentially zero.
The situation is worse in cases such as formula 1 motor racing
considered below, which requires us to consider ${\mathcal O}(100)$
competitors.

\subsection{Some simple cases}

We consider the smallest non-trivial cases $n=3$ and $n=4$.  For $n=3$
we have Plackett-Luce strengths $\alpha x,\alpha x^2,\alpha x^3$ [the
  extra factor of $x$ allows us to identify the $r$ in ``$\alpha x^r$''
  with the rank of the competitor, the best being rank 1].  Suppose
the focal competitor came first, the likelihood function would be

\begin{equation}
  \mathcal{L}(x,r)=\frac{x^{r-1}}{1+x+x^2}\qquad r=1,2,3
\end{equation}

If the focal competitor came second we would have two Plackett-Luce
probabilities to add, corresponding to different order statistics.
$\alpha x^i$ or $\alpha x^j$ coming first where $i,j\neq r$.  For
example, if $r=1$ we would have two finishing orders that are
consistent with the observation: $2\succ 1\succ 3$ and $3\succ 1\succ
2$.  These are disjoint so the likelihood will be proportional to

$$
\frac{x^2}{x+x^2+x^3}\cdot\frac{x}{x+x^3} + 
\frac{x^3}{x+x^2+x^3}\cdot\frac{x}{x+x^2}
$$

After simplification:

\begin{equation}
  \mathcal{L}(x,r)=
  \begin{cases}
    \frac{x^1}{x^1+x^2+x^3}\left(\frac{x^2}{x^1+x^3} + \frac{x^3}{x^1+x^2}\right)\qquad r=1\\
    \frac{x^2}{x^1+x^2+x^3}\left(\frac{x^1}{x^2+x^3} + \frac{x^3}{x^1+x^2}\right)\qquad r=2\\
    \frac{x^3}{x^1+x^2+x^3}\left(\frac{x^1}{x^2+x^3} + \frac{x^2}{x^1+x^3}\right)\qquad r=3
\end{cases}
\end{equation}

\begin{figure}[t]
  \begin{centering}
\includegraphics[width=4in]{123all}  % 123all.pdf produced by exponential_BT.Rmd
\caption{Log-likelihood functions \label{123all} for support
  $\in\left\lbrace 1, 2,3\right\rbrace\times\mathbb{R}$}
\end{centering}
\end{figure}


The general case with $n$ competitors is harder as we potentially have
$(n-1)!$ orderings for each observation.  Taking $n=4$ as an example,
if the focal competitor comes third we have


\newcommand{\plackett}[4]{
\frac{x^{#1}}{x^{#1} + x^{#2} + x^{#3} + x^{#4}}\cdot
\frac{x^{#2}}{         x^{#2} + x^{#3} + x^{#4}}\cdot
\frac{x^{#3}}{                  x^{#3} + x^{#4}}}

\begin{eqnarray}
\mathcal{L}(x;r=1)
\quad &=& \quad\hphantom{+}
            \plackett{2}{3}{1}{4}\nonumber\\
&{}&\quad + \plackett{2}{4}{1}{3}\nonumber\\
&{}&\quad + \plackett{3}{2}{1}{4}\nonumber\\
&{}&\quad + \plackett{3}{4}{1}{2}\nonumber\\
&{}&\quad + \plackett{4}{2}{1}{3}\nonumber\\
&{}&\quad + \plackett{4}{3}{1}{2}
\end{eqnarray}

%\begin{eqnarray}
%\mathcal{L}(x;r=2) 
%\quad &=& \quad\hphantom{+}
%            \pl{1}{3}{2}{4}\\
%&{}&\quad + \pl{1}{4}{2}{3}\\
%&{}&\quad + \pl{3}{1}{2}{4}\\
%&{}&\quad + \pl{3}{4}{2}{1}\\
%&{}&\quad + \pl{4}{1}{2}{3}\\
%&{}&\quad + \pl{4}{3}{2}{1}
%\end{eqnarray}

and so on; see figure~\ref{1234third}.


\begin{figure}[t]
  \begin{centering}
\includegraphics[width=4in]{1234third}  % 1234third.pdf produced by
                                        % exponential_BT.Rmd
\caption{Support (log-likelihood) function for the observation that
  the focal competitor came third out of four, showing
  $r=1,2,3,4$\label{1234third}.  Value of $r$ indicated in colour by
  its curve; dotted line shows (asymptotic) maximum likelihood
  estimate, viz $r=3, x\longrightarrow 0$.  Gray line at $y=-2$ shows
  the two-units-of-support limit; solutions of $\mathcal{S}=-2$
  indicated by short vertical gray lines.  For example, looking at the
  red line, we may reject the hypothesis $r=2,
  x\leqslant\exp(-1.93)\simeq 0.147$ (in favour of $r=3, x=0^+$).
  Observe that all four lines meet at $x=1$, at which point all four
  competitors have identical Plackett-Luce strengths and this
  observation is uninformative as to the value of $r$}
\end{centering}
\end{figure}

\section{Conclusions and further work}

Here a simplification of the Plackett--Luce likelihood functions for
order statistics was proposed, some simple properties obtained, and
inference problems from three disciplines given.  The results seem to
be difficult to obtain any other way.  The assumptions furnish a
reasonable, intuitive, and computationally tractable likelihood
function with wide applicability.  Further work might include the
analysis of more than one focal competitor.




%\bibliographystyle{plainat}
\bibliography{simplified.bib}



\end{document}
