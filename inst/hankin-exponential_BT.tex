\documentclass[article]{ajs}  % need to download ajs.cls, oesg.png, ajs.bst from the Austrian Journal of Statistics github repo
\RequirePackage{amsthm,amsmath,amsfonts,amssymb}
\RequirePackage{graphicx}%% uncomment this for including figures

%%%%%%%%%%%%%%%%%%%%%%%%%%%%%%
%% declarations for jss.cls %%%%%%%%%%%%%%%%%%%%%%%%%%%%%%%%%%%%%%%%%%
%%%%%%%%%%%%%%%%%%%%%%%%%%%%%%
 
%% almost as usual
\author{Robin K. S. Hankin\,\orcidlink{0000-0001-5982-0415}\\
  University of Stirling\\ Scotland}
\title{A Simplified Plackett-Luce Likelihood Function for Rank Orders}

%% for pretty printing and a nice hypersummary also set:
\Plainauthor{Robin K. S. Hankin} %% comma-separated
\Plaintitle{A Parametrized Plackett-Luce Likelihood Function for Rank Orders}


%% an abstract and keywords
\Abstract{

Here a new simplification of the Plackett--Luce likelihood functions
for order statistics is proposed.  We consider a field of $n$ entities
that are ordered in some way [the preferred example is competitors in
  a running race, ordered by time of crossing the finishing line].  We
assign Plackett--Luce strengths $\alpha x,\alpha x^2\ldots,\alpha x^n$
where $0\leq x\leqslant 1$ is a common ratio and $\alpha$ a
normalising constant.  One entity is of particular interest to us: the
``focal entity'', or focal competitor whose strength is $\alpha x^r$
for some unknown value of $r$ with $1\leqslant r\leqslant n$.  We
consider the following inference problem: given the order statistic
for the focal competitor [first, second, etc], and the total number of
competitors, what might we infer about $x$ and $r$?  Such problems
arise when non-focal competitors' performances are censored, unknown,
or irrelevant.  }

\Keywords{keywords, comma-separated, non-capitalized, \proglang{R}}
\Plainkeywords{keywords, comma-separated, non-capitalized, R} %% without formatting
%% at least one keyword must be supplied

%% publication information
%% NOTE: Typically, this can be left commented and will be filled out by the technical editor
%% \Volume{50}
%% \Issue{9}
%% \Month{June}
%% \Year{2012}
%% \Submitdate{2012-06-04}
%% \Acceptdate{2012-06-04}
%% \setcounter{page}{1}
\Pages{1--xx}

%% The address of (at least) one author should be given
%% in the following format:
\Address{
  Robin K. S. Hankin\\
  University of Stirling\\
  Scotland\\
  E-mail: \email{hankin.robin@gmail.com}\\
  URL: \url{https://www.stir.ac.uk/people/1966824}
}
%% It is also possible to add a telephone and fax number
%% before the e-mail in the following format:
%% Telephone: +43/512/507-7103
%% Fax: +43/512/507-2851

%% for those who use Sweave please include the following line (with % symbols):
%% need no \usepackage{Sweave.sty}

%% end of declarations %%%%%%%%%%%%%%%%%%%%%%%%%%%%%%%%%%%%%%%%%%%%%%%


\begin{document}

%% include your article here, just as usual

\section{Introduction}

Ranked data arises whenever multiple entities are sorted by some
criterion; the preferred interpretation is a race in which the
sufficient statistic is the order of competitors crossing the
finishing line.  The Plackett--Luce likelihood
function~\citep{luce1959,plackett1975} is a generalization of the
Bradley--Terry model~\citep{bradley1952} in which non-negative
strengths are assigned to each competitor; without loss of generality,
if the finishing order is $1,2,\ldots,n$ and competitor $i$ has
strength $p_i\geqslant 0$, then the Plackett--Luce likelihood function
for this observation will be proportional to

\begin{equation}\label{plackettluce}
\prod_{i=1}^n\frac{p_i}{\sum_{j=i}^np_j}
\end{equation}

\noindent where the strengths are normalised so $\sum p_i=1$.
However, estimation of the strengths is somewhat difficult if $n$ is
even moderately large: even with as few as 20 competitors following
Zipf's law~\citep{zipf1949} [a typical assumption for Bradley-Terry
  strengths] and say 20 independent complete rank observations,
\cite{hankin2017_rmd} shows that maximum likelihood strengths
identify the correct ranking with a probability of essentially zero.
The situation is worse in cases such as formula 1 motor racing
considered below, which requires us to consider ${\mathcal O}(100)$
competitors.

In many situations, we are not interested in the $n-1$-dimensional
simplex of BT strengths, viz $\left\lbrace(p_1,\ldots,p_n)\colon
p_i\geqslant 0,\sum p_i=1\right\rbrace$, but instead wish to make
inferences about one particular entity.  For example, we might be
interested in whether Mercedes's apparent decline in performance in
the 2023-4 season could be due to chance.  Further examples are given
in previous analysis by the same author \citep{hankin2010,
  hankin2017_rmd, hankin2020, hankin2023_formula1points}.  In these
cases, we are interested in one of the competitors and assess a number
of hypotheses about that competitor; the strengths of the others are
nuisance parameters.  We designate this entity as the {\em focal
 competitor}.

If we have $n$ competitors, optimizing the Bradley--Terry model over
$n-1$ degrees of freedom is, as noted above, a formidable task even
for moderate $n$; this suggests that some form of dimension-reducing
parametrization might be useful.  One option might be to assign the
focal competitor a BT strength proportional to $a\in [0,1]$ and
$b=1-a$ to each of $n-1$ competitors; the normalization constant would
be $n+(n-1)a$.  However, this option does not seem to be
realistic~\citep{kuenzer2025}.

The next natural suggestion would be to have $n$ competitors with
strengths $\beta, \beta x,\ldots \beta x^{n-1}$ with $0 < x\leqslant
1$.  The unit sum constraint gives $\beta=\frac{1-x}{1-x^n}$ but this
constant of proportionality cancels out in Bradley-Terry likelihoods.
The focal competitor has strength $\beta x^{r}$ for some (unknown) $r$
with $1\leqslant r\leqslant n$.  Thus $r=1$ means that the focal
competitor is the strongest among the $n$, $r=2$ means he is the
second, and so on.

\subsection{Theoretical results}

A Plackett-Luce observation is of the form $p_{(1)}\succ
p_{(2)}\succ\cdots\succ p_{(n)}$, where $p_{(i)}$ denotes the
competitor in place $i$ for $1\leqslant i\leqslant n$.  If competitor
$j$ has strength $\beta x^{j}$, then the probability of observing
$p_1\succ p_2\succ\cdots\succ p_n$ for fixed $n$ will be, according to
equation~\ref{plackettluce},

\begin{equation}
f(x)=\frac{1}{1+\cdots+x^{n-1}}\cdot
\frac{1}{1+\cdots+x^{n-2}}\cdots
\frac{1}{1+x^{2}}\cdot\frac{1}{1+x}=
\frac{1}{[n]_x!}
\end{equation}

(here $[n]_x!$ is the $q$-factorial).  We see that $f(0)=1$, $f'(x)<0$
and $f(1)=\frac{1}{n!}$; so small $x$ corresponds to a high
probability of observing the ``natural order'' in which competitor $j$
places $j$-th, $1\leqslant j\leqslant n$.  Further, we observe that
$f(x)=1-(n-1)x + {\mathcal O}(x^2)$.  If $x=1/2$, it can be shown that
the probability of observing the natural order is asymptotically $\sim
2^{-n}$.  It is interesting to note that the probability of observing
the reverse order, that is $p_n\succ p_{n-1}\succ\cdots\succ p_1$, is
$f(1/x)$ for $x>0$.

The rest of this paper considers the following inference problem: We
consider $n$ competitors with Bradley-Terry strengths proportional to
$x^i, 1\leqslant i\leqslant n$.  Our object of inference is the focal
competitor with strength $\propto x^r$, with $r$ unknown.  The
competitors are ranked randomly according to a Plackett-Luce model; we
observe only the rank of the focal competitor.  What may we infer
about $(x,r)$?

\subsection{Some simple cases}

We consider the smallest non-trivial cases $n=3$ and $n=4$.  For $n=3$
we have Plackett-Luce strengths $\alpha x,\alpha x^2,\alpha x^3$ [the
  extra factor of $x$ allows us to identify the $r$ in ``$\alpha
  x^r$'' with the nominal rank of the competitor, the best being rank
  1].  Suppose the focal competitor came first, the likelihood
function would be

\begin{equation}
  \mathcal{L}(x,r)=\frac{x^{r-1}}{1+x+x^2}\qquad r=1,2,3.
\end{equation}

If the focal competitor came second we would have two Plackett-Luce
probabilities to add, corresponding to different order statistics.
For example, if $r=1$ we would have two finishing orders that are
consistent with the observation: $2\succ 1\succ 3$ and $3\succ 1\succ
2$.  These are disjoint so the likelihood will be proportional to

$$
\frac{x^2}{x+x^2+x^3}\cdot\frac{x}{x+x^3} + 
\frac{x^3}{x+x^2+x^3}\cdot\frac{x}{x+x^2}
$$

After simplification:

\begin{equation}\label{aftersim}
  \mathcal{L}(x,r)=
  \begin{cases}
    \frac{x^1}{x^1+x^2+x^3}\left(\frac{x^2}{x^1+x^3} + \frac{x^3}{x^1+x^2}\right)\qquad r=1\\
    \frac{x^2}{x^1+x^2+x^3}\left(\frac{x^1}{x^2+x^3} + \frac{x^3}{x^1+x^2}\right)\qquad r=2\\
    \frac{x^3}{x^1+x^2+x^3}\left(\frac{x^1}{x^2+x^3} + \frac{x^2}{x^1+x^3}\right)\qquad r=3
\end{cases}
\end{equation}

\begin{figure}[t]
  \begin{centering}
\includegraphics[width=4in]{123all}  % 123all.pdf produced by exponential_BT.Rmd
\caption{The case $n=3$.  Three log-likelihood
  functions \label{123all}, one for each of the three possible
  observations of the rank $r$ of the focal competitor, presented as a
  function of $\left\lbrace 1, 2,3\right\rbrace\times\mathbb{R}^-$.
  (a), $r=1$; (b), $r=2$; (c), $r=3$}
\end{centering}
\end{figure}


The general case with $n$ competitors is harder as we potentially have
$(n-1)!$ orderings for each observation.  Taking $n=4$ as an example,
if the focal competitor comes third we have


\newcommand{\plackett}[4]{
\frac{x^{#1}}{x^{#1} + x^{#2} + x^{#3} + x^{#4}}\cdot
\frac{x^{#2}}{         x^{#2} + x^{#3} + x^{#4}}\cdot
\frac{x^{#3}}{                  x^{#3} + x^{#4}}}

\begin{eqnarray}
\mathcal{L}(x;r=1)
\quad &=& \quad\hphantom{+}
            \plackett{2}{3}{1}{4}\nonumber\\
&{}&\quad + \plackett{2}{4}{1}{3}\nonumber\\
&{}&\quad + \plackett{3}{2}{1}{4}\nonumber\\
&{}&\quad + \plackett{3}{4}{1}{2}\nonumber\\
&{}&\quad + \plackett{4}{2}{1}{3}\nonumber\\
&{}&\quad + \plackett{4}{3}{1}{2}\label{requalfour}
\end{eqnarray}

%\begin{eqnarray}
%\mathcal{L}(x;r=2) 
%\quad &=& \quad\hphantom{+}
%            \pl{1}{3}{2}{4}\\
%&{}&\quad + \pl{1}{4}{2}{3}\\
%&{}&\quad + \pl{3}{1}{2}{4}\\
%&{}&\quad + \pl{3}{4}{2}{1}\\
%&{}&\quad + \pl{4}{1}{2}{3}\\
%&{}&\quad + \pl{4}{3}{2}{1}
%\end{eqnarray}

and so on; see figure~\ref{1234third}. 

\begin{figure}[t]
  \begin{centering}
\includegraphics[width=4in]{1234third}  % 1234third.pdf produced by
                                        % exponential_BT.Rmd
\caption{Support (log-likelihood) function for the observation that
  the focal competitor came third out of four, showing
  $r=1,2,3,4$\label{1234third}.  Value of $r$ indicated in colour by
  its curve; dotted line shows (asymptotic) maximum likelihood
  estimate, viz $r=3, x\longrightarrow 0$.  Gray line at $y=-2$ shows
  the two-units-of-support limit; solutions of $\mathcal{S}=-2$
  indicated by short vertical gray lines.  For example, looking at the
  red line, we may reject the hypothesis $r=2,
  x\leqslant\exp(-1.93)\simeq 0.147$ (in favour of $r=3, x=0^+$).
  Observe that all four lines meet at $x=1$, at which point all four
  competitors have identical Plackett-Luce strengths and this
  observation is uninformative as to the value of $r$}
\end{centering}
\end{figure}

\section{The case of unknown $r$}


Although figure~\ref{1234third} shows a perfectly respectable
log-likelihood function, in practice we often do not know the value of
$r$, and need a joint likelihood function for $(r,x)$.  It is
convenient to normalize $r$ and use $s=r/n$ for consistency between
different values of $n$, and to treat $s$ as a continuous variable.
The only reasonable way to calculate such a likelihood function
appears to be numerical; fortunately this is straightforward.

We leverage the {\tt hyper2} R package~\citep{hankin2017_rmd} which
includes extensive numerical functionality to simulate Plackett-Luce
situations, and in this case function {\tt rrace()} simulates a race
between competitors of known Plackett-Luce strengths.  To obtain a
likelihood function for a specific value of $(x,s)$ at a particular
observed rank $r$, we run repeated {\em in-silico} races between
competitors of Plackett-Luce strengths $x^1,x^2,\ldots x^n$, and
observe the frequency with which competitor $\lceil ns\rceil$ places
$j$.  For example, consider $x=0.1, j=4$: each competitor has 10 times
the strength of the next, and the focal competitor placed fourth.  We
might simulate this as follows:

\label{rsession}
\begin{verbatim}
n <- 9 # number of competitors
       # observation: focal competitor places fourth

v <- seq_len(n)
rrank(15, p=0.1^v, as.character(v))
#> A ranktable:
#>       c1 c2 c3 c4 c5 c6 c7 c8 c9
#>  [1,] 2  1  4  3  5  6  7  8  9 
#>  [2,] 1  2  3  5  4  6  7  9  8 
#>  [3,] 1  2  3  4  5  6  7  8  9 
#>  [4,] 1  2  3  4  5  6  7  8  9 
#>  [5,] 1  2  4  3  5  6  7  9  8 
#>  [6,] 1  2  3  4  5  6  7  8  9 
#>  [7,] 1  3  2  4  5  6  7  8  9 
#>  [8,] 1  2  3  4  5  7  6  8  9 
#>  [9,] 3  1  2  5  4  9  6  7  8 
#> [10,] 1  2  3  4  5  6  7  8  9 
#> [11,] 1  2  3  4  5  6  7  8  9 
#> [12,] 1  2  3  5  4  6  7  9  8 
#> [13,] 1  3  2  4  5  6  7  8  9
#> [14,] 1  2  3  4  5  7  6  8  9 
#> [15,] 1  2  5  4  4  6  8  7  9 
\end{verbatim}

Above we see a bespoke {\tt hyper2} object, of class {\tt ranktable},
designed to display rank data for use with Bradley-Terry likelihoods.
We see 15 independent Plackett-Luce races; for example the first one
indicates that competitor 2 came first, 1 came second, 4 came third,
and so on.  Looking at the fourth column [headed \verb+c4+, ``came
  fourth''], we see that

\begin{equation}
  \mathcal{L}(0.1,3)\propto 1,\qquad
  \mathcal{L}(0.1,4)\propto 11,\qquad
  \mathcal{L}(0.1,5)\propto 3
\end{equation}

The maximum likelihood estimate for $s$ would therefore be $4/9\simeq
0.44$, conditional on $x=0.1$.  To estimate $x$ we would need repeated
observations, and here we note that small values of $x$ lead to
underdispersed results.

To validate the suggested likelihood method, we set $x=0.85$ and
$r=3$, that is, competitor $i$ has strength $\propto 0.85^i,
1\leqslant i\leqslant 9$; the focal competitor has $r=3$, that is, has
the third strongest Plackett-Luce strength of $0.85^3\simeq 0.61$.
The results of running {\tt rrace()} 100 times with $n=9$ are given in
table~\ref{results100insilico}.

\begin{table}
  \centering
  \begin{tabular}{llllllllll}
    place & 1  & 2  &  3 &  4 &  5 &  6 &  7 & 8 &  9  \\
    count & 14 & 13 & 18 & 12 & 13 &  9 & 11 & 8 &  2
  \end{tabular}
  \caption{Results of repeated \label{results100insilico}
    Plackett-Luce racing with $x=0.85, r=3$; thus the focal competitor
    is the third strongest.  We see that the focal competitor placed
    first 14 times, second 13 times, and so on}
\end{table}

Now, we can run repeated simulations like those on
page~\pageref{rsession}, but with $n=10^4$ trials; as support we use
$x$ running from 0.1 to 1 and $r=1,2,\ldots 9$.
Figure~\ref{toyexample} gives a visual representation of the
log-likelihood.  We see reasonable agreement between the true values
of $x$ and $r$, indicated with a red dot, and the estimates.  Note the
odd shape of the contours and the non-independence of the parameters.

\begin{figure}[t]
  \begin{centering}
\includegraphics[width=4in]{toyexample}  % toyexample.pdf produced by exponential_BT.Rmd
\caption{Contours of log-likelihood for $x$ and $r$.  Red point shows
  true values of parameters, blue dot shows maximum likelihood\label{toyexample}
  estimate}
\end{centering}
\end{figure}



\section{Formula 1}

Formula 1 racing is an important and prestigious motor sport
\citep{codling2017,jenkins2010}.  In a typical season, ${\sim} 20$
competitors race at ${\sim} 20$ venues and here we consider the 2023
season, using Lewis Hamilton as the focal competitor.  His results are
shown in table~\ref{lewishamilton}; Qatar and the US are disregarded
as he did not finish.

\begin{table}
  \centering
  \begin{tabular}{lllllllll}
venue & BHR & SAU & AUS & AZE & MIA & MON & ESP & CAN\\
place & 5   & 5   & 2   & 6   & 6   & 4   & 2   & 3  \\ \\
venue & AUT & GBR & HUN & BEL & NED & ITA & SIN & JPN\\ 
place & 8   & 3   & 4   & 4   & 6   & 6   & 3   & 5  \\ \\
venue & MXC & SAP & LVG & ABU \\
place & 2   & 8   & 7   & 9
\end{tabular}
  \caption{Lewis Hamilton's results for the 2023 season. \label{lewishamilton}
    We see that he finished fifth in Bahrain, fifth in Saudi Arabia, second in Austria, and so on}
\end{table}

Again using numerical simulation, amounting to repeated evaluation of
generalised version of equations \ref{aftersim} and \ref{requalfour}
(this time with $b=10^4$ trials per point), we obtain
figure~\ref{formula1}.  From this, we see that $s$ is less than about
0.25 and $x$ is between about 0.4 and 0.8.  Recall that, in this
model, $x=0.5$ implies that the Plackett-Luce strength of each
competitor is about twice that of the next.

\begin{figure}[t]
  \begin{centering}
\includegraphics[width=4in]{formula1}  % formula1.pdf produced by exponential_BT.Rmd
\caption{Contours of log-likelihood functions \label{formula1}.  Red circle shows
  maximum likelihood estimate}
\end{centering}
\end{figure}


\section{Conclusions and further work}

Here a parametrized Plackett-Luce likelihood function for order
statistics was proposed, some simple properties obtained, and
inference problems from motor sport given.  The results seem to
be difficult to obtain any other way.  The assumptions furnish a
reasonable, intuitive, and computationally tractable likelihood
function with wide applicability.  Further work might include the
analysis of more than one focal competitor.


%\bibliographystyle{plainat}
\bibliography{simplified.bib}



\end{document}
