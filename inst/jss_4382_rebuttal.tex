\documentclass[12pt]{article}
\usepackage{xcolor}
\begin{document}

\section*{JSS4832: Generalized Plackett-Luce Likelihoods by Robin
K. S. Hankin}


Below, the reviewer's comments are in black, and the replies to the
issues are in \textcolor{blue}{blue}.  I have indicated changes to
the manuscript where appropriate.

Short story: I have accommodated all the comments with rewording and
clarification.

\subsection*{Reviewer's comments, 15 April 2023}



Dear Robin K.S. Hankin,

We have reached a decision regarding your submission to Journal of
Statistical Software, #4832 “Generalized Plackett-Luce Likelihoods”.

Unfortunately we have to tell you that we decided to reject in its current state.

However, we encourage you to resubmit the article after detailed
revision. Please check the journal management system for any
additional feedback:
https://www.jstatsoft.org/index.php/jss/authorDashboard/submission/4832. We
expect you to provide a point-by-point response to this feedback in
your revision.

Your sincerely,

The editorial team

The manuscript describes functionality of the hyper2 package for
Fitting and testing variants of the Placket-Luce ranking model.

1.1 Recommendation

The manuscript may be publishable in the journal after revisions have been
made.

1.2 General comments

G1 The introduction should provide an overview of the contents of the manuscript.
\textcolor{blue}{Response here}\\


In addition, the section headlines could be made more informative. Currently, they are:
2. "Some simple cases"
3. "Mann-Whitney test"
4. "Multiple competitors: javelin"
5. "Formula 1 motor racing: the constructors' championship"



Maybe the headlines could provide more of a summary of what the section
contains.
\textcolor{blue}{Response here}\\ \\



G2 There should be a comparison with other software. Although the author
states in the introduction that with hyper2 "It becomes possible to analyse
a wider range of likelihood functions than standard Plackett-Luce", it is
not clear to me which of the features described later on are unique to the
hyper2 package and which can also be obtained from other packages (like,
e.g., PlackettLuce).
\textcolor{blue}{Response here}\\ \\


G3 The manuscript should comment on the timing of the functions. For the
simple example on P. 4, the maxp(H) call takes more than 8 seconds on
my computer. This makes me wonder how timings would be for realistic
examples.
\textcolor{blue}{Response here}\\ \\



1.3 Specific comments\\
S1 I'm not claiming any expertise in English language, but occasionally, I found
the manuscript difficult to understand. For example, in the abstract it
says: "The package . . . furnishes idiom" Couldn't that simply be: "The
package . . . provides functionality"? On P. 13, it says "Here we use generalized
Plackett-Luce to assess the constructors' performance". Isn't there
a "model" missing? (Plackett-Luce is just a name and cannot be used).
\textcolor{blue}{Response here}\\ \\


S2 Figure 2: For readability, maybe use xlab = expression(p[a]) and maybe
instead of a legend, write AAB, ABA, BAA next to the line in the color
of that line.
\textcolor{blue}{Response here}\\ \\


S3 P. 7: "https:github.com/RobinHankin/hyper2" should be
"https://github.com/RobinHankin/hyper2".
\textcolor{blue}{Response here}\\ \\


S4 P. 7: "Wolf" should be "Wolfe".
\textcolor{blue}{Response here}\\ \\


S5 P. 15, maxp(const2020, n=1): What does n do?
\textcolor{blue}{Response here}\\ \\



S6 P. 16, "we see strong evidence for a real decrease in the strength": I'd tone
that down. The model is almost certainly wrong, so it does not measure
the "real decrease." Also, power is unknown, so it is difficult to say what
"evidence" there is against H0. (See, e.g., Gelman, Regression an other
stories, Chapter 4: Time to unlearn what you thought you knew about
statistics. https://statmodeling.stat.columbia.edu/2022/01/27/regression-
and-other-stories-free-pdf/)
\textcolor{blue}{Response here}\\ \\



S7 The references are lacking their DOIs.
\textcolor{blue}{Response here}\\ \\

2 Review of software

The package includes many functions with a similar name. E.g., ordervec2supp,
ordervec2supp3, ordervec2supp3a, ordertable2supp, ordertable2supp3. Maybe
there could be some overview of which to use for what purpose.

\textcolor{blue}{Response here}\\ \\

Section editor comments:

The author proposes an extension of the alreadty existing hyper2 package.
This new S3 class allows for some extended likelihood specifications.

\textcolor{blue}{Response here}\\ \\


Even though this is just a code snippet and the hyper2 package was already
published elsewhere, the author should provide some introductory elaborations
on the Plackett-Luce model. Along these lines it would also be helpful to give
a quick overview of the hyper2 package, and mention other related packages. As it
is now it is difficult for a reader who's not super familiar with PL-models and
hyper2, respectively, to follow the paper. Anyway, I find the runners example
as motivating example very illustrative, and actually like the examples throughout
the paper.

\textcolor{blue}{Response here}\\ \\

Some more aesthetic effort should be made in the code file (e.g., providing
comments). The code looks like a simple purl() extraction.

\textcolor{blue}{Response here}\\ \\


\textcolor{blue}{Summary: this referee's insightful and
  constructive comments have led to a much improved submission.}

\end{document}

