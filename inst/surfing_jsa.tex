\documentclass{article}
\usepackage{amssymb}
\usepackage{graphicx}
\begin{document}
\bibliographystyle{plain}

\title{Analysis of professional surfing tournaments with generalized Bradley-Terry}


\section{Introduction}

Surfing is a popular and growing activity in terms of both
participation and viewing \cite{warshaw2010}.  Competitive surfing
involves surfers competing against one or two other surfers in heats
lasting 30-50 minutes \cite{booth1995}.  Points are awarded on a
10-point scale: each of five judges awards points to a surfer riding a
particular wave.  The World Surf League (WSL) is the main governing
body of surfing competitions \cite{wsl}.  The WSL conducts a world tour in
which the best ranked 34 surfers compete; in addition, at each tour
venue, two "wildcard" surfers also enter the competition, who are
ignored here.  The 2019 WSL tour saw competitive events held at eleven
different surf locations.

The typical format is as follows.  Up to four surfers---the
competitors---are in the water simultaneously, watched by up to five
judges.  Subject to surfing etiquette, competitors are free to catch a
wave at their discretion within the heat's time window, typically
20-30 minutes.  The judges rate each wave ride and award points to the
surfer based on a range of subjective criteria such as stylish
execution and transition of manoeuvres, but also credit elements such
as the difficulty and novelty of the performance.  The top score and
bottom score among the five judges are removed and the remaining three
judges' scores are averaged to give the final score for the surfer for
that wave.  Each surfer's aggregate score is given by the average of
the two highest-scoring waves.  The winner is the surfer with the
highest aggregate score.

Such scoring systems are designed to account for the random nature of
wave quality while reflecting competitors' abilities fairly.
Differences in wave quality between successive days mean that direct
comparison of scores between one day and another are not informative
about competitors' abilities as they are strongly dependent on details
of wave quality at the time of the competition.  However, for a
particular heat, the order statistic---that is, which competitor
scored most highly, second highest, and third---is informative about
competitors' abilities: the wave environment is common to each surfer.

We note in passing that `wct8`, hosted at Lemoore had a different
format from the other venues in the championship, being held in an
artificial wave pool in which every wave was essentially identical (up
to a mirror reflection).  Statistically, there were seven heats of
either 2, 4, 8, or 16 surfers.  We show below how to incorporate the
information present in such observations with the remainder of the
championship using a consistent and intuitive statistical model.

\subsection{Previous related work}

\cite{farley2013} analyse surfing results using one-way ANOVA techniques and
report comparisons between the top 10 and bottom 10 surfers in the
2013 World Championship Tour and conclude that the top-ranked athletes
are more consistent than the lower-ranked group. They report that "To
date, only a limited number of research studies have reported on the
results of competitive surfing" [p44], and the analysis presented here
is hoped to change that.

Other statistical research into competitive surfing appears to focus
on the scoring of aerial manoeuvres compared with other manoeuvres in
events.  \cite{undgren2014}, for example, report that aerial manoeuvres
scored higher than other stunts but had a lower completion rate, a
theme we return to below.

\section{Bradley-Terry}

The Bradley-Terry model \cite{bradley1954} assigns non-negative strengths
$p_1,\ldots, p_n$ to each of $n$ competitors in such a way that the
probability of $i$ beating $j\neq i$ in pairwise competition is
$\frac{p_i}{p_i+p_j}$; it is conventional to normalize so that $\sum
p_i=1$.  Further, we use a generalization due to \cite{luce1959}, in which
the probability of competitor~$i$ winning in a field of $\left\lbrace
1,\ldots, n\right\rbrace$ is $\frac{p_i}{p_1+\cdots +p_n}$.  Noting
that there is information in the whole of the finishing order, and not
just the first across the line, we can follow \cite{plackett1975} and
consider the runner-up to be the winner among the remaining
competitors, and so on down the finishing order. Without loss of
generality, if the order of finishing were $1,2,3,4,5$, then a
suitable Plackett-Luce likelihood function would be

\begin{equation}\label{competitors_1_to_5_likelihood}
\frac{p_1}{p_1+p_2+p_3+p_4+p_5}\cdot
\frac{p_2}{p_2+p_3+p_4+p_5}\cdot
\frac{p_3}{p_3+p_4+p_5}\cdot
\frac{p_4}{p_4+p_5}\cdot
\frac{p_5}{p_5}
\end{equation}

and this would be a forward ranking Plackett-Luce model in the
terminology of \cite{johnson2020}.  We now use a technique due to Hankin
(2010, 2017) and introduce fictional (reified) entities whose nonzero
Bradley-Terry strength helps certain competitors or sets of
competitors under certain conditions.  The original example was the
home-ground advantage in football.  If players (teams) $1,2$ with
strengths $p_1,p_2$ compete, and if our observation were $a$ home wins
and $b$ away wins for team $1$, and $c$ home wins and $d$ away wins
for team~$2$, then a suitable likelihood function would be

\[
\left(\frac{p_1+p_H}{p_1+p_2+p_H}\right)^a
\left(\frac{p_1}{p_1+p_2+p_H}\right)^b
\left(\frac{p_2+p_H}{p_1+p_2+p_H}\right)^c
\left(\frac{p_2+p_H}{p_1+p_2+p_H}\right)^d,
\]

\noindent where $p_H$ is a quantification of the beneficial home
ground effect.  Similar techniques have been used to account for the
first-move advantage in chess, and here we apply it to assess the
supposed Brazilian preference for reef-breaking waves.




\section{Likelihood-based systems and points-based systems}


\begin{table}
\begin{tabular}{l|l|l|l|l|l|l|l|l|l|l|l|l|l}
competitor  & nationality&wct01  &wct02  &wct03  & \ldots &wct09&wct10&wct11&points\\ \hline
 Ferreira   &   BRA      &   1   &   5   &   17  & \ldots      &  2   &   1   &   1   &   59740\\
 Medina     &   BRA      &   5   &   5   &   17  & \ldots      &  9   &   9   &   2   &   56475\\
 Smith 	    &   ZAF      &   3   &   3   &   17  & \ldots      &  9   &   2   &   17  &   49985\\
 Toledo     &   BRA      &   9   &   2   &   5   & \ldots      &  17  &   5   &   17  &   49145\\
 Andino     &   USA      &   2   &   17  &   5   & \dots       &  5   &   5   &   9   &   46655
\end{tabular}
\caption{WCT tour results, conventional league table\label{resultstable}}
\end{table}

Table~\ref{resultstable} shows the official results table from the
tour.  We see that Ferreira places first with 59740 points, Medina
second with 56475, and so on.  The overall ranking of the surfers is
determined by summing the season's points.  However, this system
suffers from the defect that the points system itself is intrinsically
arbitrary: the details of the points system does not affect the
competitors' behaviour but can change the overall ranking.
Plackett-Luce likelihood functions are not subject to such a
phenomenon and furnish an objective and coherent method for ranking
competitors.  Further, likelihood-based methods offer a further
advantage over points-based systems: the ability to conduct
statistically rigorous tests of meaningful nulls.  

\section{The dataset and likelihood function}

\begin{table}
\begin{verbatim}
wct01 Colapinto Bailey Wright
wct01 Freestone Lau Smith
wct01 Dora Ferreira Slater
wct01 Duru Toledo Ibelli
wct01 Moniz Heazlewood Wilson
...
wct11 Medina Florence
wct11 Colapinto Bourez
wct11 Ferreira Slater
wct11 Medina Colapinto
wct11 Ferreira Medina
\end{verbatim}
\caption{Raw results\label{rawresults}}
\end{table}

The first line of table \ref{rawresults} shows that, at one of the
heats in {\tt wct01}, Colapinto came first, Bailey second, and Wright
came third.  The last line shows that, at {\tt wct11}, Ferreira came
first and Medina second.  We can convert this dataset into a
Plackett-Luce likelihood function but first have to remove the
wildcards and also player Vries whose maximum likelihood strength is
zero.  For the first line we would have Plackett-Luce likelihood function as

\begin{equation}
\frac{p_\mathrm{Colapinto}}{p_\mathrm{Colapinto} + p_\mathrm{Bailey} + p_\mathrm{Wright}}\cdot
\frac{p_\mathrm{Bailey}}{p_\mathrm{Bailey} + p_\mathrm{Wright}}
\end{equation}

[observe that there is no requirement here for the strengths to have
unit sum].  The entire dataset has a likelihood function that
incorporates all 477 lines, but it includes:

\begin{equation}
\frac{
p_\mathrm{Andino}^{19}\, p_\mathrm{Callinan}^{19}\, p_\mathrm{Coffin}^{8}\ldots
}{
(p_\mathrm{Andino} + p_\mathrm{Bourez})  (p_\mathrm{Andino} + p_\mathrm{Buchan})(p_\mathrm{Bourez} + p_\mathrm{Flores} + p_\mathrm{Freestone})\ldots
}
\end{equation}

\begin{figure}
\includegraphics[width=4in]{surfing_files/figure-html/showsurfingmaxppie-3.png} % Pie chart of strengths
\caption{Pie chart\label{piechartmax} showing maximum likelihood strengths of the 23 competitors.  Note the
dominance of Ferreira, Florence, and Medina}
\end{figure}

Figure~\ref{piechartmax} shows the maximum likelihood estimate for the
23 competitors' strengths, which appears to show a wide range from
Florence at about 17.5\% down to Coffin at about 1.5\%.  The {\tt
hyper2} software may be used to formally assess one plausible null:
that the competitors all have equal skill, that is $H_0\colon
p_\mathrm{Andino} = p_\mathrm{Bourez}=\ldots=
p_\mathrm{Wright}=\frac{1}{23}$, and any differences in placings are
due to random variation.  The Method of Support \cite{edwards1973}
rejects the null (in favour of the evaluate) at $p=4\times 10^{-5}$.
There is thus strong evidence to suggest that the competitors are
indeed of different skills and the result is not random.  Note that
the competitors' points (that is, the final column of
table~\ref{resultstable}) does not admit this type of analysis, as the
points awarded are arbitrary.

However, points awarded may be compared with likelihood estimates, and
one would expect high points totals to be associated with high
likelihoods.  Figure~\ref{compare_likelihood_points} shows a loose
correlation ($R^2=0.64$; $p=3\times 10^{-6}$) and
Figure~\ref{compare_likelihood_points_rankings} shows how the rankings
differ when calculated by the points and likelihood systems.  Note the
anomalous position of Florence, who did not compete in five venues due
to injury: this strongly affecting his points total but not his
likelihood-based ranking.

\begin{figure}[h]
\includegraphics[width=4in]{surfing_files/figure-html/directlogplotofstrengths-1.png}
\caption{Scatterplot of points scored  \label{compare_likelihood_points} 
against maximum likelihood estimates}
\end{figure}

\begin{figure}[h]
\includegraphics[width=4in]{surfing_files/figure-html/ordertransplotplot-1.png}
\caption{Scatterplot of ranks \label{compare_likelihood_points_rankings} of points-based ranking
against likelihood-based ranking.  Note the anomalous position of
Florence}
\end{figure}



\bibliography{chess}
\end{document}
