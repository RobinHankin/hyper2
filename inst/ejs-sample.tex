\documentclass[ejs]{imsart}

%% Packages
\RequirePackage{amsthm,amsmath,amsfonts,amssymb}
\RequirePackage[numbers]{natbib}
%\RequirePackage[authoryear]{natbib}%% uncomment this for author-year citations
\RequirePackage[colorlinks,citecolor=blue,urlcolor=blue]{hyperref}
\RequirePackage{graphicx}

\arxiv{2010.00000}
\startlocaldefs
%%%%%%%%%%%%%%%%%%%%%%%%%%%%%%%%%%%%%%%%%%%%%%
%%                                          %%
%% Uncomment next line to change            %%
%% the type of equation numbering           %%
%%                                          %%
%%%%%%%%%%%%%%%%%%%%%%%%%%%%%%%%%%%%%%%%%%%%%%
%\numberwithin{equation}{section}
%%%%%%%%%%%%%%%%%%%%%%%%%%%%%%%%%%%%%%%%%%%%%%
%%                                          %%
%% For Axiom, Claim, Corollary, Hypothesis, %%
%% Lemma, Theorem, Proposition              %%
%% use \theoremstyle{plain}                 %%
%%                                          %%
%%%%%%%%%%%%%%%%%%%%%%%%%%%%%%%%%%%%%%%%%%%%%%
\theoremstyle{plain}
\newtheorem{axiom}{Axiom}
\newtheorem{claim}[axiom]{Claim}
\newtheorem{theorem}{Theorem}[section]
\newtheorem{lemma}[theorem]{Lemma}
%%%%%%%%%%%%%%%%%%%%%%%%%%%%%%%%%%%%%%%%%%%%%%
%%                                          %%
%% For Assumption, Definition, Example,     %%
%% Notation, Property, Remark, Fact         %%
%% use \theoremstyle{definition}            %%
%%                                          %%
%%%%%%%%%%%%%%%%%%%%%%%%%%%%%%%%%%%%%%%%%%%%%%
\theoremstyle{definition}
\newtheorem{definition}[theorem]{Definition}
\newtheorem*{example}{Example}
\newtheorem*{fact}{Fact}
%%%%%%%%%%%%%%%%%%%%%%%%%%%%%%%%%%%%%%%%%%%%%%
%%                                          %%
%% For Case use \theoremstyle{remark}       %%
%%                                          %%
%%%%%%%%%%%%%%%%%%%%%%%%%%%%%%%%%%%%%%%%%%%%%%
\theoremstyle{remark}
\newtheorem{case}{Case}
%%%%%%%%%%%%%%%%%%%%%%%%%%%%%%%%%%%%%%%%%%%%%%
%% Please put your definitions here:        %%
%%%%%%%%%%%%%%%%%%%%%%%%%%%%%%%%%%%%%%%%%%%%%%
\endlocaldefs

\begin{document}
\begin{frontmatter}
\title{A simplified Plackett-Luce likelihood function for rank orders}
\runtitle{Simplified Plackett-Luce likelihood functions}
%\thankstext{T1}{A sample additional note to the title.}

\begin{aug}
%%%%%%%%%%%%%%%%%%%%%%%%%%%%%%%%%%%%%%%%%%%%%%%
%% Only one address is permitted per author. %%
%% Only division, organization and e-mail is %%
%% included in the address.                  %%
%% Additional information can be included in %%
%% the Acknowledgments section if necessary. %%
%% ORCID can be inserted by command:         %%
%% \orcid{0000-0000-0000-0000}               %%
%%%%%%%%%%%%%%%%%%%%%%%%%%%%%%%%%%%%%%%%%%%%%%%
\author[A]{\fnms{Robin}~\snm{Hankin}\ead[label=e1]{hankin.robin@gmail.com}},
%%%%%%%%%%%%%%%%%%%%%%%%%%%%%%%%%%%%%%%%%%%%%%
%% Addresses                                %%
%%%%%%%%%%%%%%%%%%%%%%%%%%%%%%%%%%%%%%%%%%%%%%
\address[A]{Computer Science and Mathematics,
University of Stirling\printead[presep={,\ }]{e1}}

\end{aug}

\begin{abstract}

Here a simplification of the Plackett-Luce likelihood functions for
order statistics is proposed.  We consider a field of $n+1$ entities
that are ordered in some way [the preferred example is competitors in
  a running race, ordered by time of crossing the finishing line].
One entity is of particular interest to us: the ``focal entity'', or
focal competitor.  We assign Placket-Luce strength of $a$ to the focal
competitor and $b$ to all others; we require $a+b=1$.  This device
furnishes a reasonable, intuitive, and computationally tractable
likelihood function that I apply to a range of applications including
Formula 1 motor racing and the International Mathematical Olympiad.
The results are difficult to obtain with other methods.
\end{abstract}

\begin{keyword}[class=MSC]
\kwd[Primary ]{6202}
\kwd{00X00}
\kwd[; secondary ]{62F30}
\end{keyword}

\begin{keyword}
\kwd{First keyword}
\kwd{second keyword}
\end{keyword}

\end{frontmatter}
%%%%%%%%%%%%%%%%%%%%%%%%%%%%%%%%%%%%%%%%%%%%%%
%% Please use \tableofcontents for articles %%
%% with 50 pages and more                   %%
%%%%%%%%%%%%%%%%%%%%%%%%%%%%%%%%%%%%%%%%%%%%%%
%\tableofcontents

\section{Introduction}

Ranked data arises whenever multiple entities are sorted by some
criterion; the preferred interpretation is a race in which the
sufficient statistic is the order of competitors crossing the
finishing line.  The Plackett-Luce likelihood
function~\cite{luce1959,plackett1975} is a generalization of the
Bradley-Terry~\cite{bradley1952} in which non-negative strengths are
assigned to each competitor; without loss of generality, if the
finishing order is $1,2,\ldots,n$ and competitor $i$ has strength
$p_i\geqslant 0$, then the Plackett-Luce likelihood function will be
proportional to

\begin{equation}\label{plackettluce}
\prod_{i=1}^n\frac{p_i}{\sum_{j=i}^np_i}
\end{equation}

\noindent where the strengths are normalised so $\sum p_i=1$.
However, estimation of the strengths is somewhat difficult if $n$ is
even moderately large: even with as few as 20 competitors following
Zipf's law~\cite{zipf1949} [a typical assumption for Bradley-Terry
  strengths], Hankin~\cite{hankin2017_rmd} shows that maximum
likelihood strengths identify the correct ranking with a probability
of essentially zero.  The situation is worse in cases such as the
International Mathematical Olympiad, which requires us to consider
over 100 competitors.

Consider a race between a competitor of Bradley-Terry strength $a$ and
$n$ cloned competitors, each of strength $b=1-a$.  An observation is
indexed by~$r$, the number of strength $b$ clones finishing ahead
of~$a$, and because we have $n+1$ competitors $0\leqslant r\leqslant
n$.  The initial field strength is $a+nb=n+(n-1)a$.  The Plackett-Luce
likelihood function~\cite{plackett1975,luce1959} for this situation
would be

\begin{equation}\label{likeforrn}
  \mathcal{L}_r(a)=\begin{cases}
\frac{b}{a+ n   b}\cdot
\frac{b}{a+(n-1)b}\cdot
\frac{b}{a+(n-2)b}\cdots
\frac{a}{a+(n-r)b}\qquad &{r<n}\\
\frac{b}{a+ n   b}\cdot
\frac{b}{a+(n-1)b}\cdot
\frac{b}{a+(n-2)b}\cdots
\frac{b}{a+b}\qquad &{r=n}
  \end{cases}
\end{equation}

Although these expressions do have closed-form analytical forms, it is
not helpful in numerical work because they tend to be very unstable,
especially where $kb\simeq n$ for some integer $k$, as limiting forms
have to be used.  Note carefully that the expressions in
equation~\ref{likeforrn} are likelihoods, not probabilities, and do
not sum to 1.

\begin{figure}[t]
\includegraphics[width=4in]{ninelikes}  % produced by very_simplified_likelihood.Rmd
\caption{Likelihood functions for $r=0(1)8$\label{ninelikes}}
\end{figure}

Previous related work presented the case $n=2$ [in current notation].
Considering the case $n=7$ gives us $7+1=8$ distinct likelihood
functions, Figure~\ref{ninelikes}.  We see an unanticipated asymmetry
between $r$ and $n-r$, and in particular that
$\hat{a}\geqslant\frac{r}{n}$, with equality only in the edge-cases
$r=0$ or $n$.  For $r=3$ one might expect that $\hat{a}=\frac{1}{2}$
on the grounds that the focal competitor's performance was median.
However, $\hat{a}\simeq 0.514$.

\begin{figure}[t]
\includegraphics[width=4in]{dotprobs}  % produced by very_simplified_likelihood.Rmd
\caption{Maximum likelihood estimator $\hat{a}$ for $n=1(1)10$ and $r=0(1)n$
  \label{dotprobs}}
\end{figure}

\section{Use-cases}


Formula 1 motor racing is an important and prestigious motor sport
\citep{codling2017,jenkins2010}.  Season ranking is based on a points
allocation system wherein competitors are awarded points based on race
finishing order; points accumulate additively.  The overall
competition winner is the competitor who accumulates the most points
after the final race.  However, as argued by Hankin~\cite{hankin2023},
the drivers' {\em ranks} are statistically sufficient for analysis of

The methods discussed above may be used to assess Bradley-Terry
strengths of Formula 1 racing drivers.  Let us consider 2018 season as
an example and first consider Lewis Hamilton.

Lewis Hamilton is a 


Pierre Gasly and Esteban Ocon have a long-standing rivalry and
comparisons between them typically compare disparate metrics such as
number of podiums, and highest grid position in a season.

\begin{acks}[Acknowledgments]
The authors would like to thank the anonymous referees, an Associate
Editor and the Editor for their constructive comments that improved the
quality of this paper.
\end{acks}


%%%%%%%%%%%%%%%%%%%%%%%%%%%%%%%%%%%%%%%%%%%%%%
%% Supplementary Material, including data   %%
%% sets and code, should be provided in     %%
%% {supplement} environment with title      %%
%% and short description. It cannot be      %%
%% available exclusively as external link.  %%
%% All Supplementary Material must be       %%
%% available to the reader on Project       %%
%% Euclid with the published article.       %%
%%%%%%%%%%%%%%%%%%%%%%%%%%%%%%%%%%%%%%%%%%%%%%
\begin{supplement}
\stitle{Title of Supplement A}
\sdescription{Short description of Supplement A.}
\end{supplement}
\begin{supplement}
\stitle{Title of Supplement B}
\sdescription{Short description of Supplement B.}
\end{supplement}

%%%%%%%%%%%%%%%%%%%%%%%%%%%%%%%%%%%%%%%%%%%%%%%%%%%%%%%%%%%%%
%%                  The Bibliography                       %%
%%                                                         %%
%%  imsart-???.bst  will be used to                        %%
%%  create a .BBL file for submission.                     %%
%%                                                         %%
%%  Note that the displayed Bibliography will not          %%
%%  necessarily be rendered by Latex exactly as specified  %%
%%  in the online Instructions for Authors.                %%
%%                                                         %%
%%  MR numbers will be added by VTeX.                      %%
%%                                                         %%
%%  Use \cite{...} to cite references in text.             %%
%%                                                         %%
%%%%%%%%%%%%%%%%%%%%%%%%%%%%%%%%%%%%%%%%%%%%%%%%%%%%%%%%%%%%%

%% if your bibliography is in bibtex format, uncomment commands:
\bibliographystyle{imsart-number} % Style BST file (imsart-number.bst or imsart-nameyear.bst)
\bibliography{hyper2}       % Bibliography file (usually '*.bib')

\end{document}
