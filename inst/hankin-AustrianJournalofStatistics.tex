\documentclass[article]{ajs}
\RequirePackage{amsthm,amsmath,amsfonts,amssymb}
\RequirePackage{graphicx}%% uncomment this for including figures

%%%%%%%%%%%%%%%%%%%%%%%%%%%%%%
%% declarations for jss.cls %%%%%%%%%%%%%%%%%%%%%%%%%%%%%%%%%%%%%%%%%%
%%%%%%%%%%%%%%%%%%%%%%%%%%%%%%
 
%% almost as usual
\author{Robin K. S. Hankin\,\orcidlink{0000-0001-5982-0415}\\
  University of Stirling\\ Scotland}
\title{A Simplified Plackett-Luce Likelihood Function for Rank Orders}

%% for pretty printing and a nice hypersummary also set:
\Plainauthor{Robin K. S. Hankin} %% comma-separated
\Plaintitle{A Simplified Plackett-Luce Likelihood Function for Rank Orders}


%% an abstract and keywords
\Abstract{


Here a simplification of the Plackett--Luce likelihood functions for
order statistics is proposed.  We consider a field of $n$ entities
that are ordered in some way [the preferred example is competitors in
  a running race, ordered by time of crossing the finishing line].
One entity is of particular interest to us: the ``focal entity'', or
focal competitor.  We assign Plackett--Luce strength $a$ to the focal
competitor and $b$ to all others; we require $a+b=1$.  The associated
inference problem has a wide range of applications and I present some
analysis of datasets drawn from the fields of Olympic athletics,
education, and Formula~1 motor racing.

}
\Keywords{keywords, comma-separated, non-capitalized, \proglang{R}}
\Plainkeywords{keywords, comma-separated, non-capitalized, R} %% without formatting
%% at least one keyword must be supplied

%% publication information
%% NOTE: Typically, this can be left commented and will be filled out by the technical editor
%% \Volume{50}
%% \Issue{9}
%% \Month{June}
%% \Year{2012}
%% \Submitdate{2012-06-04}
%% \Acceptdate{2012-06-04}
%% \setcounter{page}{1}
\Pages{1--xx}

%% The address of (at least) one author should be given
%% in the following format:
\Address{
  Robin K. S. Hankin\\
  University of Stirling\\
  Scotland\\
  E-mail: \email{hankin.robin@gmail.com}\\
  URL: \url{https://www.stir.ac.uk/people/1966824}
}
%% It is also possible to add a telephone and fax number
%% before the e-mail in the following format:
%% Telephone: +43/512/507-7103
%% Fax: +43/512/507-2851

%% for those who use Sweave please include the following line (with % symbols):
%% need no \usepackage{Sweave.sty}

%% end of declarations %%%%%%%%%%%%%%%%%%%%%%%%%%%%%%%%%%%%%%%%%%%%%%%


\begin{document}

%% include your article here, just as usual

\section{Introduction}

Ranked data arises whenever multiple entities are sorted by some
criterion; the preferred interpretation is a race in which the
sufficient statistic is the order of competitors crossing the
finishing line.  The Plackett--Luce likelihood
function~\citep{luce1959,plackett1975} is a generalization of the
Bradley--Terry model~\citep{bradley1952} in which non-negative
strengths are assigned to each competitor; without loss of generality,
if the finishing order is $1,2,\ldots,n$ and competitor $i$ has
strength $p_i\geqslant 0$, then the Plackett--Luce likelihood function
for this observation will be proportional to

\begin{equation}\label{plackettluce}
\prod_{i=1}^n\frac{p_i}{\sum_{j=i}^np_j}
\end{equation}

\noindent where the strengths are normalised so $\sum p_i=1$.
However, estimation of the strengths is somewhat difficult if $n$ is
even moderately large: even with as few as 20 competitors following
Zipf's law~\citep{zipf1949} [a typical assumption for Bradley--Terry
  strengths] and say 20 independent complete rank observations,
\cite{hankin2017_rmd} shows that maximum likelihood strengths
identify the correct ranking with a probability of essentially zero.
The situation is worse in cases such as formula 1 motor racing
considered below, which requires us to consider ${\mathcal O}(100)$
competitors.

\subsection{Simplified Plackett-Luce likelihoods}

As noted, likelihood function~\ref{plackettluce} becomes unmanageable
with $n$ even moderately large and some form of simplification is
needed.  To that end, consider a Plackett-Luce likelihood function for
a race between a competitor of Bradley-Terry strength $a$ and
$n-1\geqslant 1$ cloned competitors, each of strength $b=1-a$.  Under
these circumstances, an observation is indexed by~$r$, the rank of the
focal competitor, and because we have $n$ competitors $1\leqslant
r\leqslant n$.  The initial field strength is $a+(n-1)b=1+(n-2)b$.  The
Plackett--Luce likelihood function~\citep{luce1959,plackett1975} for
this situation would be

\begin{eqnarray}\label{likeforrn1}
  \mathcal{L}_{r,n}(a) &=&
\frac{b}{1+ (n-2)b}\cdot
\frac{b}{a+(n-3)b}\cdots\frac{a}{a+(n-r-1)b}\nonumber\\
&=& \frac{B-1}{(B+n-1)(B+n-2)\cdots(B+n-r-1)}\nonumber\\ 
&=& \frac{(B-1)(B+n-r-2)!}{(B+n-1)!}\label{likeforrn3}
\end{eqnarray}

\noindent where $B=1/b$.  Note carefully that these expressions are
likelihoods, not probabilities, and do not sum to 1.  Previous related
work \citep{hankin2024_hyper3} presented the case $n=3$ [in current
  notation].  Considering the case $n=8$ gives us $8+1=9$ distinct
likelihood functions, Figure~\ref{ninelikes}.  We see an unanticipated
asymmetry between $r$ and $n-r$.  For $r=5$ one might expect that
$\hat{a}=\frac{1}{2}$ on the grounds that the focal competitor's
performance was median.  However, $\hat{a}\simeq 0.5886$.
Figure~\ref{dotprobs} shows the maximum likelihood estimator $\hat{a}$
for various values of $n,r$, obtained by direct numerical maximization
of equation~\ref{likeforrn3}.

\begin{figure}[t]
  \begin{centering}
\includegraphics[width=4in]{ninelikes}  % ninelikes.pdf produced by very_simplified_likelihood.Rmd
\caption{Likelihood functions for $r=0(1)8$\label{ninelikes}.  Value
  of $r$ indicated in colour by its curve; vertical line shows
  $\hat{a}\simeq 0.5886$ for the case $r=4$}
\end{centering}
\end{figure}


\subsection{Asymptotic analysis}

Analysis of expression~\ref{likeforrn3} for large $n$ and $r=pn$ for
fixed $p$ shows that the support
$\mathcal{S}=\log\mathcal{L}_{r,n}(a)$ is asymptotically

\begin{equation}\label{asymptotic}
\mathcal{S}=
%\log\mathcal{L}_{r,n}(a)=
\log(B-1) + B\log(1-r/n)
+\mathcal{O}\left(n^{-1}\right)
\end{equation}

(up to an arbitrary additive constant).  Using
$\mathcal{S}^*=\log(B-1)+B\log(1-r/n)$ as an approximate support thus
incurs an error of only $\mathcal{O}(n^{-1})$.  Observing that
$\partial^2\mathcal{S}^*/\partial B^2=-(B-1)^{-2}<0$, the evaluate
would be unique; we have $\hat{B}=1-\log(1-r/n)^{-1}$, or
alternatively $\hat{a}=(1-\log(1-r/n))^{-1}$.  Compare this with the
result of asking ``what is the probability that the focal competitor
beats a randomly chosen opponent?'' to which the binomial maximum
likelihood estimate is $\frac{r-1}{n-r}$.  For repeated independent
observations $r_i,n_i$, $1\leqslant i\leqslant N$ we would have

\begin{equation}
  \hat{a} =   \frac{1}{1-\sum\log(1-r_i/n_i)/N};
\end{equation}

compare the binomial result of $\sum(r_i-1)/\sum n_i$.  Below, I
present three examples that showcase likelihood
function~\ref{likeforrn3}, drawn from the fields of athletics,
education, and Formula~1 motor racing.

\begin{figure}[t]
  \begin{centering}
\includegraphics[width=4in]{dotprobs}  % produced by very_simplified_likelihood.Rmd
\caption{Maximum likelihood estimator $\hat{a}$ for $n=2(1)10$
  and\label{dotprobs}} $r=1(1)n$
\end{centering}
\end{figure}

% \subsection{Olympic track and field athletics}
% 
% One long-standing contention in the athletic world is that of lane
% bias~\citep{munro2022}.  Running in the inner lane is frequently held
% to be a disadvantage; mechanisms include biomechanical issues arising
% from centrifugal acceleration, to competitors' inability to see their
% rivals.  Here attention is restricted to heats, for which lane placing
% is random.  I consider every heat in Men's 100m Olympic running from
% 2004--2020 (a total of 41 heats, each of 8 or 9 competitors); only
% competitors' ranks are used.  Figure~\ref{plotolymp} shows a
% log-likelihood curve for the Bradley--Terry strength of the focal
% competitor, here the runner in the innermost lane.  We see that the
% evaluate of $0.511$ is very close to the neutral value of $0.5$, for
% which the support is only $0.029$, far short of the standard two units
% of support criterion for rejection~\citep{edwards1972}.  This dataset
% contains no evidence to support lane bias.
% 
% \begin{figure}[t]
%   \begin{centering}
% \includegraphics[width=4in]{plotolymp} % produced by very_simplified_likelihood.Rmd
% \caption{Log-likelihood curve for the Bradley--Terry strength of the
%   lane-1 runner, Olympic qualifying heats 2012-2020 \label{plotolymp}}
% \end{centering}
% \end{figure}

\subsection{Park run}

``Parkrun'' is a distributed community initiative that organises
weekly timed 5 km runs/walks in parks worldwide~\citep{hindley2020}.
A typical event will have 200 participants.  Table~\ref{parkruntable}
shows the author has completed a total of 21 parkruns to date,
comprising 9 runs in 2023 and 12 in 2024.  From the first pair of
numbers we see that 259 runners attended that particular parkrun in
2023, of whom the author placed 177.

\begin{table}[t]
  \centering
  \caption{Parkrun results}
\label{parkruntable}
\begin{tabular}{cccccccccc}\\
 & \multicolumn{9}{c}{2023}\\
rank   & 177& 222& 206& 142& 118& 224& 128& 115& 183\\
runners& 259& 305& 297& 241& 179& 338& 203& 245& 254\\ \\
\end{tabular}
\begin{tabular}{ccccccccccccc}
  & \multicolumn{12}{c}{2024}\\
rank   &  193& 240& 197& 257& 227& 234& 238& 238& 196& 242& 242& 318\\
runners&  145& 157& 156& 191& 172& 174& 173& 165& 172& 199& 181& 229\\
\end{tabular}
\end{table}

Is there evidence that the author's performance differs between
years?  Figure~\ref{parkruncontour} shows contours of
support [again normalised so the support of the evaluate is zero] for
the parkrun dataset, as a function of 2023 strength on the
horizontal axis and 2024 strength on the vertical axis.  A null of
equal strengths appears as the diagonal line.  We see that the maximum
likelihood estimate is $\hat{p}_\mathrm{2023}=0.427$,
$\hat{p}_\mathrm{2024}=0.408$.  However, a constrained optimization
shows that the maximum support along the null diagonal is about
$-2.43$, at ${p}_\mathrm{2023}={p}_\mathrm{2024}\simeq 0.409$.  The
two units of support criterion~\citep{edwards1972} is met, and we are
justified in concluding that the student is indeed more competent at
applied than pure mathematics.  Alternatively, we may observe that
$2\mathcal{S}=4.86$ is in the tail region of its asymptotic $\chi^2_1$
distribution, corresponding to a $p$-value of about $0.028$.

Now consider table~\ref{fisherparkrun}, giving a summary of whether
nonfocal ocmpetitors place ahead of the author, split by venue.  A
two-sided Fisher test gives a $p$-value of about $5\times 10^{-14}$,
surely an unreasonable conclusion given the author's perceived
performance stagnation.


\begin{figure}[t]
  \begin{centering}
\includegraphics[width=4in]{parkruncontour} % produced by very_simplified_likelihood.Rmd
\caption{Support contours for Bradley-Terry strength
  for \label{parkruncontour} 2023 and 2024.  Null of equal strengths
  shown as a diagonal line}
\end{centering}
\end{figure}



\begin{table}[h]
    \begin{tabular}{ l c r }
       & \multicolumn{2}{c}{Faster?}\\
       & yes  & no  \\
  2023 & 1506 & 806 \\
  2024 & 2102 & 708
    \end{tabular}
    \caption{Two-way contingency\label{fisherparkrun} table}
\end{table}

\subsection{Education}

In educational research, confidentiality of examination scores is
often an issue, and individual results are generally regarded as
sensitive and private information.  Nevertheless, it is common for a
student to possess some informative observations, for example the
total enrollment in a specific class and their individual ranking
within that class.  Table~\ref{educationtable} shows an apocryphal
dataset: what could a student infer from such data?

\begin{table*}[t]
  \caption{Educational dataset}
\label{educationtable}
\begin{tabular}{llll}
\hline
                   course&category   &rank&class size\\ \hline
      rings and modules  &  pure     & 9  &      12\\
           group theory  &  pure     & 7  &      17\\
               calculus  & applied   & 2  &      23\\
         linear algebra  & applied   & 3  &       9\\
 differential equations  & applied   & 2  &      13\\
               topology  &   pure    & 8  &      14\\
     special relativity  & applied   & 3  &      13\\
        fluid mechanics  & applied   & 4  &      12\\
            Lie algebra  &   pure    & 9  &      15\\
     \hline
\end{tabular}
\end{table*}

One might wonder whether the student was better at pure or applied
mathematics and, if so, to quantify the difference.
Figure~\ref{plotpureandapplied} shows contours of support [again
  normalised so the support of the evaluate is zero] for the dataset,
as a function of pure strength on the horizontal axis and applied
strength on the vertical axis.  A null of equal strengths appears as
the diagonal line.  We see that the maximum likelihood estimate is
$\hat{p}_\mathrm{pure}=0.542$, $\hat{p}_\mathrm{applied}=0.852$.
However, a constrained optimization shows that the maximum support
along the null diagonal is about $-2.06$, at
${p}_\mathrm{pure}={p}_\mathrm{applied}\simeq 0.683$.  The two units of
support criterion~\citep{edwards1972} is met, and we are justified in
concluding that the student is indeed more competent at applied than
pure mathematics.  Alternatively, we may observe that
$2\mathcal{S}=4.11$ is in the tail region of its asymptotic $\chi^2_1$
distribution, corresponding to a $p$-value of about $0.042$.

\begin{figure}
\begin{centering}
\includegraphics[width=4in]{plotpureandapplied}  % plotpureandapplied.pdf produced by very_simplified_likelihood.Rmd
\caption{Contours of equal support for pure (horizontal) and applied
  (vertical) strength; null of equal strengths shown as a diagonal
  line \label{plotpureandapplied}}
\end{centering}
\end{figure}


\subsection{Formula 1}

Formula 1 motor racing is an important and prestigious motor sport
\citep{codling2017,jenkins2010}.  Season ranking is based on a points
allocation system wherein competitors are awarded points based on race
finishing order; points accumulate additively.  The overall
competition winner is the competitor who accumulates the most points
after the final race.  However, as argued by
\cite{hankin2023_formula1points}, the drivers' {\em ranks} are
statistically sufficient for analysis using Plackett--Luce approaches.
The methods presented here may be used to assess competitive strengths
of Formula 1 racing driver Sergio P\'{e}rez (``Checo'') over his
career from 2011 to 2023.  Checo's Formula 1 performance began
somewhat inauspiciously as second driver for Sauber for whom he
finished no higher than P7 in 2011, but his performance rose
subsequently, culminating in his achievement of a startling 20 podium
finishes in 2022--3.  Noting that Checo raced against 73 distinct
competitors---this rendering estimation of the other players'
Plackett--Luce strengths effectively
impossible~\citep{hankin2020}---we are nevertheless in a position to
consider competitive strength using the modified likelihood approach
presented here.  We consider the following inference problem:

\begin{equation}
  a = a(t) = \frac{e^{\alpha + \beta t}}{1+e^{\alpha + \beta t}},\qquad\alpha,\beta\in\mathbb{R}
\end{equation}

(that is, a logistic regression of generalized Bradley--Terry strength
$a$ against time $t$, with $\alpha$ and $\beta$ being linear
coefficients).  Contours of equal support are shown in
Figure~\ref{showchecolike}, again normalized so that
$\mathcal{S}(\hat{\alpha},\hat{\beta})=0$.  We see the expected
elliptical contours, and further $H_0\colon\beta=0$ may be rejected in
favour of $\alpha=-0.27$, $\beta=0.081$, on the grounds that
$\mathcal{S}(\alpha,0)\leqslant 9.2$.  Alternatively we may observe
that $2\mathcal{S}=18.4$ would correspond to a $p$-value of about
$1.7\times 10^{-5}$ on its asymptotic $\chi^2_1$ null distribution.

\begin{figure}[t]
  \begin{centering}
\includegraphics[width=4in]{showchecolike}  % showchecolike.pdf produced by very_simplified_likelihood.Rmd
\caption{Log-likelihood contours for $\alpha,\beta$ for Checo's racing
  career 2011--2023\label{showchecolike}}
\end{centering}
\end{figure}

\section{Conclusions and further work}

Here a simplification of the Plackett--Luce likelihood functions for
order statistics was proposed, some simple properties obtained, and
inference problems from three disciplines given.  The results seem to
be difficult to obtain any other way.  The assumptions furnish a
reasonable, intuitive, and computationally tractable likelihood
function with wide applicability.  Further work might include the
analysis of more than one focal competitor.




%\bibliographystyle{plainat}
\bibliography{hyper2}



\end{document}
